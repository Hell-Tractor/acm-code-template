% !TEX encoding = UTF-8 Unicode
\documentclass[a4paper,11pt,twoside,fontset = fandol,UTF8]{ctexbook}

\usepackage[a4paper,scale=0.8,hcentering,bindingoffset=8mm]{geometry}
\usepackage{listings} % 代码块
\usepackage[hidelinks]{hyperref}
\usepackage{color} % 颜色
\usepackage{fontspec} % 字体
\usepackage{graphicx} % 图片
\usepackage{fancyhdr} % 页眉页脚
\usepackage{lastpage}
\usepackage{afterpage}
\usepackage{xeCJK}
\usepackage{framed}
\usepackage{amsmath}

\definecolor{orange}{rgb}{1, 0.64, 0}

% 代码块属性设置
\lstset{
  language=C++,
  numbers=left,
  frame=single,
  tabsize=2,
  backgroundcolor=\color[RGB]{245, 245, 244},
  showstringspaces=false,
  basicstyle=\fontspec{Consolas},
  rulecolor=\color{black},
  commentstyle=\color[RGB]{0, 120, 0},
  keywordstyle=\color{blue}\bfseries,
  stringstyle=\color[RGB]{128, 0, 0},
  morecomment=[l][\color{red}]{//\ !},
  morecomment=[l][\color{cyan}]{//\ *},
  morecomment=[l][\color{orange}]{//\ ?},
  mathescape=true,
  breaklines=true
}

% 页码设置
\pagestyle{fancy}
\fancyhf{}
% \fancyfoot[C]{\thepage\ / \pageref{LastPage}}
\fancyhead[LE,RO]{\thepage}

\fancypagestyle{plain}{
  \fancyhead[LE,RO]{\thepage}
}

\definecolor{shadecolor}{rgb}{0.92, 0.92, 0.92}

\newcommand{\emptyPage}{\null\thispagestyle{empty}\addtocounter{page}{-1}\newpage}
%代码块设置
%param: author, desp, time, space, code
\newcommand{\code}[5]{
  \begin{itemize}
    \setlength{\itemsep}{0pt}
    \item \textbf{作者} \\ #1
    \item \textbf{描述} \\ #2
    \item \textbf{时间复杂度} \\ O($ #3 $)
    \item \textbf{空间复杂度} \\ O($ #4 $)
    \item \textbf{代码}
    \lstinputlisting{#5}
  \end{itemize}
}

\begin{document}
  \begin{titlepage}       % 封面
    \centering
    \vspace*{\baselineskip}
    \rule{\textwidth}{1.6pt}
    \vspace*{-\baselineskip}
    \vspace*{2pt}
    \rule{\textwidth}{0.4pt}\\[\baselineskip]{\LARGE CODE TEMPALTE\\[\baselineskip]\small for ACM XCPC}
    \\[0.2\baselineskip]
    \rule{\textwidth}{0.4pt}\vspace*{-\baselineskip}\vspace{3.2pt}
    \rule{\textwidth}{1.6pt}\\[\baselineskip]
    \scshape

    \begin{figure}[!htb]
        \centering
        \includegraphics[width=0.3\textwidth]{icpc}    % 当前tex文件同一目录下名为icpc的任意格式图片
    \end{figure}

    \vspace*{3\baselineskip}
    Edit by Team\\
    [\baselineskip]
    {\Large \CJKfontspec{Microsoft YaHei UI}  \par}
    {\Large Love Cut Love Cut Love \\ \normalsize{at XMU}\par}
    \vfill
    {\scshape 2022.7}\\{\large XIAMEN}\par
  \end{titlepage}

  \emptyPage

  \setcounter{page}{1}
  \tableofcontents


  \newpage
  \emptyPage

  \setcounter{page}{1}

  \chapter{图论}
  \setcounter{page}{1}
  \section{最短路}
  \subsection{Dijkstra}
  \code{Hell Tractor}{优先队列优化Dijkstra \\ {\color{red}不能处理负环与负权图(带负边的图为负权图)} \\ 负权图的处理可以借鉴Johnson全源最短路为每个点增加势能的思想}{MlogM}{N + M}{codes/dijkstra.cpp}
  \subsection{暴力Dijkstra}
  \code{Hell Tractor}{暴力Dijkstra}{N^2 + M}{N + M}{codes/dijkstra_cruel.cpp}
  \subsection{Bellman-Ford}
  \code{Hell Tractor}{求负环 \\ {\color{red}只能判定以s为起点是否能抵达负环,若要判断图中是否有负环,需要建立超级源点,并向所有其他点连一条边权为0的有向边。}}{NM}{N + M}{codes/bellman_ford.cpp}
  \subsection{Johnson}
  \code{Hell Tractor}{任意图全源最短路}{NMlogM}{N^2 + M}{codes/johnson.cpp}
  \section{Tarjan}
  \subsection{SCC}
  \code{Hell Tractor}{Tarjan求强连通分量}{N + M}{N + M}{codes/tarjan_scc.cpp}
  \subsection{割点}
  \code{Hell Tractor}{Tarjan求割点}{N + M}{N + M}{codes/tarjan_bcc.cpp}
  \subsection{桥}
  \code{Hell Tractor}{Tarjan求桥}{N + M}{N + M}{codes/tarjan_bridge.cpp}
  \subsection{圆方树}
  \code{Hell Tractor}{构建圆方树}{N + M}{N + M}{codes/RST.cpp}
  \section{网络流}
  \subsection{ISAP最大流}
  \code{Hell Tractor}{没什么好说的}{N ^ 2 M}{N + M}{codes/isap.cpp}
  \subsection{Dinic费用流}
  \code{Hell Tractor}{基于SPFA实现, f为最大流}{NMf}{N + M}{codes/dinic.cpp}
  \section{二分图匹配}
  \subsection{匈牙利算法}
  待填
  \subsection{KM算法}
  待填
  \section{欧拉(回)路}
  \code{Hell Tractor}{
    \begin{itemize}
      \item 无向图是欧拉图:当且仅当非零度顶点连通,且顶点的度都为偶数
      \item 无向图是半欧拉图:当且仅当非零度顶点连通,且恰有0或2个顶点的度为奇数
      \item 有向图是欧拉图:当且仅当非零度顶点强连通,且每个顶点的入度等于出度
      \item 有向图是半欧拉图:当且仅当非零度顶点弱连通,且至多有一个顶点的入度比出度大1,至多有一个顶点的出度比入度大1,其余顶点的入度等于出度
    \end{itemize}
  }{N + M}{N + M}{codes/hierholzer.cpp}
  \section{最小生成树}
  \subsection{Kruskal}
  \code{Hell Tractor}{没什么好说的}{MlogM}{N + M}{codes/kruskal.cpp}
  \subsection{优先队列优化Prim}
  \code{Hell Tractor}{一般情况下都使用Kruskal算法。在稠密图尤其是完全图上,暴力Prim(本算法不是暴力Prim)的复杂度比Kruskal优,但\textbf{不一定}实际跑得更快。}{(N + M)logN}{N + M}{codes/prim.cpp}
  \subsection{暴力Prim}
  \code{Hell Tractor}{暴力Prim}{N ^ 2 + M}{N + M}{codes/prim_cruel.cpp}
  
  \chapter{树}
  \section{线段树}
  \subsection{递归式线段树}
  \code{Hell Tractor}{没什么好说的}{logN}{N * 4}{codes/segmentTree.cpp}
  \subsection{zkw线段树}
  \code{Hell Tractor}{没什么好说的}{logN}{N * 4}{codes/zkw.cpp}
  \section{倍增}
  \code{Hell Tractor}{没什么好说的}{logN}{NlogN}{codes/lca.cpp}
  \section{重链剖分}
  \code{Hell Tractor}{没什么好说的}{logN}{N * k}{codes/HLD.cpp}
  \section{平衡树}
  \subsection{Treap}
  \code{Hell Tractor}{没什么好说的}{logN}{N}{codes/treap.cpp}
  \subsection{替罪羊树}
  \code{Hell Tractor}{没什么好说的}{logN}{N}{codes/ScapeGoatTree.cpp}
  \subsection{Splay}
  \code{Hell Tractor}{区间反转+区间操作}{logN}{N}{codes/splay.cpp}
  
  \chapter{其他数据结构}
  \section{ST表}
  \code{jjjjssss6}{区间最值维护}{NlogN or 1}{NlogN}{codes/st.cpp}
  \section{莫队}
  \code{jjjjssss6}{没什么好说的}{N\sqrt{N}}{N}{codes/mo.cpp}
  \section{并查集}
  \code{Hell Tractor}{没什么好说的}{N\alpha(N)}{N}{codes/ufs.cpp}

  \chapter{数论}
  \section{快速傅里叶变换}
  \code{Hell Tractor}{将多项式系数表达转换为点值表达}{NlogN}{N}{codes/FFT.cpp}
  \section{扩展欧几里得}
  \code{Hell Tractor}{求解二元一次不定方程}{Metaphysics}{1}{codes/exgcd.cpp}
  \section{逆元}
  \code{jjjjssss6}{exgcd求逆元}{Metaphysics}{1}{codes/inv.cpp}
  \section{中国剩余定理}
  \code{jjjjssss6}{没什么好说的}{N}{N}{codes/crt.cpp}
  \section{Lucas定理}
  \code{jjjjssss6}{求组合数}{N}{N}{codes/lucas.cpp}
  \section{矩阵}
  \code{Hell Tractor}{没什么好说的}{N \; or \; N ^ 2}{N}{codes/matrix.cpp}
  \section{快速幂}
  \code{Hell Tractor}{没什么好说的}{logN}{1}{codes/qpow.cpp}
  \section{高斯消元}
  \code{jjjjssss6}{线性方程组求解}{N^3}{N^2}{codes/GaussianElimination.cpp}
  \section{线性基}
  \code{jjjjssss6}{n个数选任意个使异或值最大}{N^2}{N}{codes/basis.cpp}
  \section{数论分块}
  \code{jjjjssss6}{$f(i)=\lfloor \frac{n}{i} \rfloor = v$ 时 $i$ 的取值范围是$[l, r]$}{N}{1}{codes/ntblock.cpp}
  \section{欧拉函数}
  \code{jjjjssss6}{小于x的数中与x互质的个数}{N}{N}{codes/EulerFunction.cpp}
  \section{数学公式}
  \subsection{平方和公式}
$$
\sum_{i=1}^n i^2 = \frac{n(n+1)(2n+1)}{6}
$$
\subsection{立方和公式}
$$
\sum_{i=1}^n i^3 = \left[\frac{n(n+1)}{2}\right]^2
$$
\subsection{排列组合公式}
\begin{itemize}
	\item 在 $n$ 个不同物体中,可重复的选取 $r$ 个物体的排列数为:$n^r$
	\item 在 $n$ 个不同物体中,可重复的选取 $r$ 个物体的组合数为:$C_{n+r-1}^r$
	\item 在 $A=\{1, 2, \cdots, n\}$ 中取 $r$ 个不相邻的数进行组合,其组合数为:$C_{n-r+1}^r$
	\item  $n$ 个物体中不相同的物体的总数是 $k$ 个,即 $n=\sum_{i=1}^k n_i$ ,则这几个物体全排列数是:$\frac{n!}{\prod_{i=1}^k n_i!}$
	\item 圆周排列(选取的物体不分先后):$\frac{A_n^r}{r}$
	\item 项链排列(在圆周排列的基础上,正面向上和反面向上两种方式放置个数是同一个排列):$\frac{A_n^r}{2r}$
\end{itemize}
\subsection{多重集}
在 $S$ 中任选r个元素的组合称为 $S$ 的 $r$ 组合,$\forall i$ 当 $r \leq n_i$ 时,有公式$C(n; n_1 \times a_1, n_2 \times a_2, \cdots, n_k \times a_k)=C_{k + r - 1}^r$ \\
\hspace*{0.7cm} 由公式可以看出多重集合的组合只与类别数 $k$ 和选取的元素 $r$ 有关,与总数无关。
\begin{itemize}
	\item 设元素 $a_1, a_2, \cdots, a_n$ 互不相同,从无限多重集 $\{\infty \times a_1, \infty \times a_2, \cdots, \infty \times a_n\}$ 中取 $r$ 个元素的组合数为:$C_{n + r - 1}^r$
	\item 设元素 $a_1, a_2, \cdots, a_n$ 互不相同,从有限多重集 $\{k_1 \times a_1, k_2 \times a_2, \cdots, k_3 \times a_n\}$,各 $k$ 均大于等于 $r$,取 $r$ 个元素的组合数为:$C_{n + r - 1}^r$
	\item 设元素 $a_1, a_2, \cdots, a_n$ 互不相同,从有限多重集 $\{k_1 \times a_1, k_2 \times a_2, \cdots, k_3 \times a_n\}$ 中取 $r$ 个元素,至少存在一个 $k_i < r$ 时,令生成函数 $G(x) = \sum_{j = 0}^{k_i} x^j, i = 1, 2, \cdots, n$,$G(x)$ 中 $x_r$ 的系数即为所求
\end{itemize}
\subsection{海伦公式}
设平面内有一个三角形,边长分别为 $a$,$b$,$c$,则三角形的面积 $S = \sqrt{p(p-a)(p-b)(p-c)}$,其中 $p=\frac{a+b+c}{2}$
\subsection{点到直线的距离公式}
$$
d = \frac{\left| Ax_0 + By_0 + C \right|}{\sqrt{A^2 + B^2}}
$$
\subsection{点到平面的距离公式}
$$
d = \frac{\left| Ax_0 + By_0 + Cz_0 + D \right|}{\sqrt{A^2 + B^2 + C^2}}
$$
\begin{itemize}
	\item 三角形重心:\\ 设某个三角形的重心为 $G(cx, cy)$,顶点坐标分别为$A_1(x_1, y_1)$,$A_2(x_2, y_2)$,$A_3(x_3, y_3)$,则有:
		$$
		cx = \frac{x_1 + x_2 + x_3}{3} \qquad cy = \frac{y_1 + y_2 + y_3}{3}
		$$
	\item 多边形重心: \\
		$$
		cx = \frac{\sum_i cx_i \times s_i}{3 \times \sum_i s_i}
		\qquad
		cy = \frac{\sum_i cy_i \times s_i}{3 \times \sum_i s_i}
		$$
		其中,$(cx_i, cy_i)$,$s_i$分别是所划分的第$i$个三角形的重心和面积
\end{itemize}
\subsection{斯特林公式}
$$
n! \approx \sqrt{2 \pi n}\left(\frac{n}{e}\right)^n
$$
更加准确的极限
$$
\lim_{n \rightarrow \infty} \frac{e^n n!}{n^n \sqrt{n}} = \sqrt{2 \pi}
\quad OR \quad
\lim_{n \rightarrow \infty} \frac{n!}{\sqrt{2 \pi n} \left(\frac{n}{e}\right)^n} = 1
$$
\subsection{范德蒙德卷积}
$$
\sum_{i = 0}^k \binom{n}{i} \binom{m}{k - i} = \binom{n + m}{k}
$$
\subsection{汉诺塔}
$$
2^{n - 1}
$$
\subsection{基姆拉尔森公式(计算某天是星期几)}
$$
w = \left( d + 2m + 3 \left( \frac{m + 1}{5} \right) + y + \left( \frac{y}{4} \right) - \left( \frac{y}{100} \right) + \left( \frac{y}{400} \right) + 1 \right) \% 7
$$
\subsection{卡特兰数}
应用:进出栈问题、排队方式、二叉树生成问题、凸多边形三角形划分、括号匹配问题、满二叉树个数、圆划分问题、填充问题
\begin{itemize}
	\item 递推公式1 \\
		$$
		f(n) = \sum_{i = 0}^{n - 1} f(i) \times f(n - i - 1)
		$$
	\item 递推公式2 \\
		$$
		f(n) = \frac{2(2n - 1)f(n - 1)}{n + 1}
		$$
	\item 组合公式1 \\
		$$
		f(n) = \frac{C_{2n}^n}{n + 1}
		$$
	\item 组合公式2 \\
		$$
		f(n) = C_{2n}^{n} - C_{2n}^{n - 1}
		$$
	\item 通项公式 \\
		$$
		C_n = \binom{2n}{n} - \binom{2n}{n - 1}  = \frac{\binom{2n}{n}}{n + 1} = \frac{C_{n-1} (4n - 2)}{n + 1}
		$$
	\item 生成函数 \\
		$$
		C(x) = \frac{1 - \sqrt{1 - 4x}}{2x}
		$$
\end{itemize}
\subsection{贝尔数}
$B_n$ 表示将 $n$ 个集合划分成若干个非空集合的方案数
\begin{itemize}
	\item 递推公式 \\
		$$
		B_n = \sum_{i = 0}^{n - 1} \binom{n - 1}{i} B_i
		$$
	\item 通项公式 \\
		$$
		B_n = \frac{1}{e} \sum_{i \geq 0} \frac{i^n}{i!}
		$$
\end{itemize}
\subsection{分拆数}
将 $n$ 进行整数分拆的方案数 $O(n\sqrt{n})$
$$
F_n = \left\{
	\begin{aligned}
		& 0 & n < 0 \\
		& 1 & n = 0 \\
		& \sum_{k \geq 1} (-1)^{k + 1} \left(F_{n - \frac{k(3k - 1)}{2}} + F_{n - \frac{k(3k + 1)}{2}}\right) & n > 0
	\end{aligned}
	\right.
$$
\subsection{康托展开}
当前排列在全排列中的名次问题
$$
X = \sum_{i = 1}^n a_i \times (i - 1)!
$$
\subsection{旋转同构}
$n$ 个点,每个点移动 $k(0 \leq k \leq n - 1)$ 步,循环个数为:$gcd(k, n)$
\subsection{循环同构}
圆上有 $n$ 个点,$k$ 种颜色,旋转同构,求总方案数
$$
ans = \frac{1}{n}\sum_{i = 0}^{n - 1} k \times gcd(i, n)
$$
\subsection{对称同构}
圆上有 $n$ 个点,$k$ 种颜色,旋转同构、翻转同构,求总方案数 \\
\hspace*{0.7cm} 旋转同构总共形成 $a = \sum_{i = 0}^{n - 1} k^{gcd(i, n)}$ 个不动点 \\
\hspace*{0.7cm} 翻转同构分两种情况考虑:
\begin{itemize}
	\item 当\textbf{ $n$ 为奇数时},对称轴有 $n$ 条,每条对称轴形成 $\frac{n - 1}{2}$ 个长度为 $2$ 的循环,$1$ 个长度为 $1$ 的循环(对称轴一定过一个顶点),共 $\frac{n - 1}{2} + 1 = \frac{n + 1}{2}$ 个循环。 \\ 因此总共形成 $b = nk^{\frac{n + 1}{2}}$ 个不动点。
	\item 当\textbf{ $n$ 为偶数时},有两种对称轴。穿过两个点的对称轴有 $\frac{n}{2}$,共形成 $\frac{n}{2} + 1$ 个循环;不穿过点的对称轴有 $\frac{n}{2}$,共形成 $\frac{n}{2}$ 个循环。 \\ 因此总共形成 $b = \frac{n}{2}\left(k^{\frac{n}{2} + 1} + k^{\frac{n}{2}}\right)$ 个不动点。
\end{itemize}
\hspace*{0.7cm} 综上, $ans = \frac{a + b}{2n}$


  \chapter{字符串}
  \section{KMP}
  \code{Hell Tractor}{字符串匹配}{NlogN}{N}{codes/kmp.cpp}
  \section{AC自动机}
  \code{jjjjssss6}{AC自动机加topo优化}{N}{N}{codes/ac_automation.cpp}
  \section{PMT}
  \code{jjjjssss6}{最长公共前缀}{N}{N}{codes/pmt.cpp}
  \section{Manacher}
  \code{jjjjssss6}{马拉车算法}{N}{N}{codes/manacher.cpp}
  \section{SAM}
  \subsection{狭义SAM}
  \code{jjjjssss6}{狭义SAM}{N}{N}{codes/SAM.cpp}
  \subsection{广义SAM}
  \code{jjjjssss6}{广义SAM}{N}{N}{codes/exSAM.cpp}
  \section{SA}
  \code{jjjjssss6}{后缀数组}{N}{N}{codes/string_SA.cpp}
  
  \chapter{计算几何}
  \section{向量}
  \subsection{平面向量}
  \code{Hell Tractor}{没什么好说的}{1}{1}{codes/vector2D.cpp}
  \section{凸包}
  \subsection{二维凸包}
  \code{Hell Tractor}{没什么好说的}{NlogN}{N}{codes/convexHull.cpp}
  
  \chapter{杂项}
  \section{代码头}
  \lstinputlisting{codes/header.cpp}
  \section{模拟退火}
  \code{Hell Tractor}{
    使用时派生SA类,重写函数J, getNewState, checkState(可选重写)以及构造函数 \\
    并需要设定type为所需求值类型
  }{Metaphysics}{1}{codes/SA.cpp}
  \section{IO}
  \code{Hell Tractor}{fread读入优化}{1}{1}{codes/IO.cpp}
  \section{高精度}
  待填
  \section{分数类}
  \code{Hell Tractor}{{\color{orange}未经测试的代码!谨慎使用}}{1}{1}{codes/Fraction.cpp}

\end{document}
\label{LastPage}